\documentclass{article}
\usepackage[pdftex]{graphicx}
\usepackage{amsmath}

\pdfpagewidth 8.5in
\pdfpageheight 11in
\topmargin -1in
\headheight 0in
\headsep 0in
\textheight 8.5in
\textwidth 6.5in
\oddsidemargin 0in
\evensidemargin 0in
\headheight 77pt
\headsep 0in
\footskip .75in

\title{CS1510  Dynamic Programming Problems 10 \& 11}
\author{Rebecca Negley, Sean Myers}
\date{September 28, 2011}

\begin{document}
\maketitle

\begin{enumerate}
\setcounter{enumi}{12}
\item Our goal is now to consider the Knapsack problem, and develop a method for computing the actual items to be taken in $O(L)$ space and $O(nL)$ time.
\begin{enumerate}
\item Consider the following problem. The input is the same as for the knapsack problem, a collection of $n$ items $I_{1}, ..., I_{n}$ with weights $w_{1}, ..., w_{n}$ and values $v_{1}, ..., v_{n}$ and a weight limit $L$. The output is in two parts. First you want to compute the maximum value of a subset $S$ of the $n$ items. that has weight at most $L$, as well as the weight of this subset. Let us call this value and weight $v_{a}$ and $w_{a}$. Secondly for this subset $S$ you want to compute the weight and value of the  items in $\{I_{1}, ..., I_{n/2}\}$ that are in $S$. Let us call this value and weight $v_{b}$ and $w_{b}$. So your output  will be two weights and two values. Give an algorithm for this problem that uses space $O(L)$ and time $O(nL)$.
\newline
\newline Solution: Consider the algorithm used in number 12 where we construct a size $2\times L$ matrix. Here, we will modify that algorithm to construct a $2\times L\times 3$ matrix. The third matrix dimension corresponds to 3 values: the current value, the value that the current set had at $I_{n/2}$, and the weight that the current set had at $I_{n/2}$. Then, the space is $O(L)$. Before item $n/2$, set the $I_{n/2}$ weight and value to zero. We loop through all $n$ inputs and rotate between the two rows of the matrix rather than traversing down an $n\times L$ matrix. After the $n/2$ iteration, we copy the $n/2$ value and weight from the location we use to get our current value. Without these modifications, our algorithm is essentially the one described in class. At each iteration $i$, the max value at weight $j$ is the maximum of the maximum value at weight $j$ in the $i-1$ iteration and the maximum value at weight $j-w_i$ in the $i-1$ iteration + $v_i$, corresponding to whether we add the $i$th item or not. Then, the algorithm would look like:
\begin{tabbing}
kna\= p($I_1$,\= ,\ldots \= ,$I_n$, $L$)\\
\>ks[$2][L$]\\
\>for $i$ from $1$ to $\frac{n}{2}-1$\\
\>\>for $j$ from $1$ to $L$\\
\>\>\>if \= $w_i<j$\\
\>\>\>\>ks$[i\%2][j][0]=\max \{$ks$[(i-1)\%2][j][0],$ks$[(i-1)\%2][j-w_i][0]+v_i\}$\\
\>\>\>else\\
\>\>\>\>ks$[i\%2][j][0]=$ks$[(i-1)\%2][j][0]$\\
\>\>\>ks$[i\%2][j][1]=0$\\
\>\>\>ks$[i\%2][j][2]=0$\\
\>for $j$ from $1$ to $L$\\
\>\>if $w_{n/2}<j$\\
\>\>\>ks$[0][j][0]=\max \{$ks$[1][j][0],$ks$[1][j-w_i][0]+v_i\}$\\
\>\>else\\
\>\>\>ks$[0][j][0]=$k$[1][j][0]$\\
\>\>ks$[0][j][1]=$ks$[0][j][0]$\\
\>\>ks$[0][j][1]=j$\\
\>for $i$ from $\frac{n}{2}+1$ to $n$\\
\>\>for $j$ from $1$ to $L$\\
\>\>\>if  $w_i<j$\\
\>\>\>\>ks$[i\%2][j][0]=\max \{$ks$[(i-1)\%2][j][0],$ks$[(i-1)\%2][j-w_i][0]+v_i\}$\\
\>\>\>else\\
\>\>\>\>ks$[i\%2][j][0]=$ks$[(i-1)\%2][j][0]$\\
\>\>\>if $w_i<j$ AND ks$[(i-1)\%2][j-w_i][0]+v_i>$ks$[i\%2][j][0]$\\
\>\>\>\>ks$[i\%2][j][1]=$ks$[(i-1)\%2][j-w_i][1]$\\
\>\>\>\>ks$[i\%2][j][2]=$ks$[(i-1)\%2][j-w_i][2]$\\
\>\>\>else\\
\>\>\>\>ks$[i\%2][j][1]=$ks$[(i-1)\%2][j][1]$\\
\>\>\>\>ks$[i\%2][j][2]=$ks$[(i-1)\%2][j][2]$\\
\>$\{v_a,w_a\}=\{\max \limits_{j}^{} \ $ks$[L\%2][j][0],j\}$\\
\>$\{v_b,w_b\}=\{$ks$[L\%2][w_a][1],$ks$[L\%2][w_a][2]\}$\\
\>return $\{v_a,w_a,v_b,w_b\}$
\end{tabbing}
\ 


\item Explain how to use the algorithm from the previous subproblem to get a divide and conquer algorithm for finding the items in the Knapsack problem a and uses space $O(L)$ and time $O(nL)$. 
\newline
\newline Solution: Observe that the algorithm above gives us the maximum value and and its corresponding weight and the value and weight that the optimal set had at step $n/2$. We know that the items in $\{I_1,\ldots ,I_{n/2}\}$ that are in the optimal solution have total weight $w_b$ and total value $v_b$. Then, we must have that the items in $\{I_{n/2+1},\ldots ,I_n\}$ that are in the optimal solution have weight $w_a-w_b$ and value $v_a-v_b$.  We can then break this problem down into two subproblems: knap($I_1,\ldots ,I_{n/2},w_b$) and knap($I_{n/2+1},\ldots ,I_n,w_a-w_b$).  Consider a divide a concquer algorithm where we follow through the algorithm above and then call knap($I_1,\ldots ,I_{n/2},w_b$) and knap($I_{n/2+1},\ldots ,I_n,w_a-w_b$) recursively to construct our optimal subset. Add the following base case to knap($I_j,\ldots ,I_k$,val): if $j=k$ and $val>0$, return $I_j$. Otherwise, return the empty set. The runtime of the algorithm in part (a) $n*L$. Then, runtime(knap($I_1,\ldots ,I_n,L$)) $ = $ runtime(knap($I_1,\ldots ,I_{n/2},L_1$))+runtime(knap($I_{n/2+1},\ldots ,I_n,L_2$))+ n*L, where$ L_1+L_2\le L$. Observe $\frac{n}{2}*L1+\frac{n}{2}*L2\le \frac{n}{2}L$, so runtime$=O(n*L)$.
\newline
\end{enumerate}


\setcounter{enumi}{15}
\item Give an algorithm for the following problem whose running time is polynomial in $n+W$:
\newline Input: positive integers $w_{1}, ..., w_{n}$, $v_{1}, ..., v_{n}$, and $W$
\newline Output: The maximum possible value of $\sum\limits_{i=1}^{n}x_{i}*v_{i}.$ subject to $\sum\limits_{i=1}^{n}x_{i}*w_{i}. \le W$ and each $x_{i}$ is a \newline nonnegative integer.


Solution: This problem will create a branching factor of $W$ at each node in the tree, because we know that $x_i$ has to be less than $W$. To solve this problem, we need to make sure that at each level, even if the branching factor is W, it only creates W branches at each sublevel. We need some pruning rules to do this:

\begin{enumerate}
\item If we have chosen $x_1 ... x_{i-1}$, prune any branches where $x_i * w_i + \sum\limits_{j=0}^{i-1}w_{j}*x_{i} > W$. In other words, prune any branches where adding the next weight will make the overall weight larger than $W$.
\item If two branches $b_1$ and $b_2$ have weights where $w_{b1} = w_{b2}$, and both $b_1$ and $b_2$ are on the same level $L$, prune the tree with lesser sum of values.
\end{enumerate}

The second rule makes sense because if you have a correct solution with the lower summed value: $v_{b1} + v_{x1}+...+v_{xn}$, if it is the same weight then it can be replaced and have a higher value, making the above solution an incorrect solution, since it is not the max value.

With these two pruning rules, let's now create the algorithm. First, we make a table maxSum[n, W], and initialize all the values to 0. Each index will hold the max possible value for that level and that corresponding weight. 

With this initialization, we know that row 0 is correct because nothing has been put in the set. So our first for loop will go from 1 to n. This will correspond to the depth of the tree we are at. Call this loop lvl for level

At each level, we must go through every weight, so we need another for loop going from 0 to W. call it curWeight. We also need to try all possible values of x, that equal up to W in each index, so we need yet another for loop of x going from 0 to the current iteration of curWeight, above.

First we must check if this solution is feasible, so we must check the constraints of the problem. If  x*weight[lvl] $>$ curWeight, then we continue because it violates pruning rule 1. Otherwise, take the max of either the current value at position maxSum[lvl,curWeight] or the value[lvl]*x + maxSum[lvl-1, curWeight-weight[lvl]*x]. In the second case, it is equivalent to saying: grab me the optimal value of the current weight minus the weight of the current thing we are trying to add. If the current value is greater than this, do not add it, otherwise add it. 

In pseudo code, it will look like this:
 
\begin{tabbing}
for\=(i = 0 to n):\\
\>for\=(j = 0 to W):\\
\>\>maxSum[i][j] = 0; //Initalization \\
\\
for(lvl = 0 to n):\\
\>for(curWeight = 0 to W):\\
\>\>for\=(x = 0 to curWeight):\\
\>\>\>if(x*weight[lvl] $>$ curWeight): continue;\\
\>\>\>maxSum[lvl,curWeight] = max(maxSum[lvl, curWeight], value[lvl]*x + maxSum[lvl, curWeight-weight[lvl]*x])\\
\end{tabbing}

At most x will be W, so let's say that for every inner loop, x is W iterations. Then the final runtime will be $O(n*w^2)$, which is a polynomial time of n+W. More specifically, $(n+W)^3$.

\item Give an algorithm for the following problem whose running time is polynomial in $n+L$ where 
\newline $L= max(\sum\limits_{i=1}^{n}v_{i}^{3}, \prod\limits_{i=1}^{n}v_{i} )$
\newline Input: positive integers $v_{1}, ..., v_{n}$
\newline Output: A subset $S$ of the integers such that $\sum_{v_{i} \in S} v_{i}^{3} = \prod_{v_{i}\in S} v_{i}$

Solution: First, it is recognizeable that this will be a binary tree, where each level is a branch whether you take the coin or not (and combinations of previous coin taking). The leaves will produce the solution if there is one. There are going to be $2^n$ leaves without pruning, which is not acceptable. So let's prune!

The only pruning rule that is applicable is if there are two values such that $v_i = v_j$, and are on the same level, then prune one. Another possibility would be: $v_{i1}^3 + ... +v_i^3  = v_{j1}^3 +...+v_{j}^3 $ or $v_{i1}^3 * ... *v_i^3  = v_{j1} *...*v_{j}$, or in layman's terms: if there are some values that are sum to the same value or multiply to the same value, and are on the same level, prune one. For if I get two answers of 80 on level $k$, and one of them leads to the answer at the leaf level, for example $80*4*3$ or $80 + 4^3 + 3^3$, then interchanging the values to get to 80 does not matter.

If we set up an array of boolean values like subAlg[lvl, S, M] where S is the summation of all the values, and M is the multiplicative product of all the values. First we set all values in the array to 0, except for 0,0,0 which we set to 1. 

First we run a for loop on the levels, from 0 to n. In each , level, we transfer the table from the current level to the lower level. To do so, we must have for loops running through all the S and M. To compute the lower level, you set the top level 1's to the positions of subAlg[lvl-1, S, M] or the sum and multiplication if you don't accept the value of the lower level, value[lvl-1]. If you accept it, you change the value of both $S+value[lvl-1]^3$ and M*value[lvl-1] to 1 as well, or subAlg[lvl-1, $S+value[lvl-1]^3$,M*value[lvl-1]]. 

In code it looks like this:
\begin{tabbing}

initialize subAlg[n,S,M] array. to all 0's\\
subAlg[0,0,0] = 1\\
\\
for\=(lvl = 0 to n):\\
\>for\=(i= 0 to S):\\
\>\>for\=(j = 0 to M):\\
\>\>\>if(value[i+1]*j > M || $value[i+1]^3$ > S): continue;\\
\>\>\>subAlg[lvl+1,i,j] = subAlg[lvl,i,j]\\
\>\>\>subAlg[lvl+1,i+$value[i+1]^3$, j*value[lvl+1]] = subAlg[lvl,i,j]\\
\end{tabbing}

The final algorithm will run in n*L*L, which is polynomial $n+L$. Finding the subset, is looking at subAlg[n][p][p], where p goes from 1 to min(S,M). If one of them is 1, then there is a subset, and then backtrack to find the values that were included.
\end{enumerate}
\end{document}
