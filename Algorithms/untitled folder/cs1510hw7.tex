\documentclass{article}
\usepackage[pdftex]{graphicx}
\usepackage{amsmath}
\usepackage[T1]{fontenc}
\usepackage{courier}
\usepackage[table]{xcolor}
\pdfpagewidth 8.5in
\pdfpageheight 11in
\topmargin -1in
\headheight 0in
\headsep 0in
\textheight 8.5in
\textwidth 6.5in
\oddsidemargin 0in
\evensidemargin 0in
\headheight 77pt
\headsep 0in
\footskip .75in

\title{CS1510 Dynamic Algorithms}
\author{Sean Myers}
\date{September 14, 2011}

\begin{document}
\maketitle

\begin{enumerate}
\item Give an efficient algorithm for finding the shortest common super-sequence of two strings $A$ and $B$. $C$ 
\newline is a super-sequence of $A$ if and only if $A$ is a subsequence of $C$.

Solution:
\begin{tabbing}
SCS(\= char A[m], char B[n]):\\
\>int SCS[m+1,n+1]\\
\>for(\=i = 0 to n):\\
\>\>SCS[0,n] = n\\
\>for(i = 1 to m):\\
\>\>for(\=j = 1 to n):\\
\>\>\>if(\=A[m] == B[n]):\\
\>\>\>\>SCS[m,n] = SCS[m-1, n-1] +1\\
\>\>\>else:\\
\>\>\>\>SCS[m,n] = min(SCS[m-1,n] +1, SCS[m,n-1]+1)\\
\>return SCS[m,n] //Length of Shortest Common Super-Sequence
\end{tabbing}

\item Consider the algorithm that you developed for the previous problem.
\newline
\begin{enumerate}
\item Show the table that your algorithm constructs for the inputs $A = zxyyzz$, and $B = zzyxzy$

\begin{tabular}{| l | c | c | c | c | c | c | r | }
\hline
   	& * & \textit{z} & \textit{x} & \textit{y} & \textit{y} & \textit{z} & \textit{z} \\ \hline
  * 	   & 0 & 1 & 2 & 3 & 4 & 5 & 6 \\ \hline
\textit{z} & 1 & 1 & 2 & 3 & 4 & 5 & 6 \\ \hline
\textit{z} & 2 & 2 & 3 & 4 & 5 & 5 & 6 \\ \hline
\textit{y} & 3 & 3 & 4 & 4 & 5 & 6 & 7 \\ \hline
\textit{x} & 4 & 4 & 4 & 5 & 6 & 7 & 8 \\ \hline
\textit{z} & 5 & 5 & 5 & 6 & 7 & 7 & 8 \\ \hline
\textit{y} & 6 & 6 & 6 & 6 & 7 & 8 & 9 \\ \hline
\end{tabular}


\item Explain how to find the length of the shortest common super-sequence in your table.

The strings that were used to get the answer will have lengths m and n. Do an array query on row m and column n to get the optimal length.
\newline

\item Explain how to compute the actual shortest common super-sequence from your table by tracing back from the table entry that gives the length of the shortest common super-sequence. 
\newline

To explain this, first let us look at what all array searches will be comparing:

\begin{tabular}{ l | c}
$x_{1}$ & $x_{2}$ \\ \hline
$x_{3}$ & $x_{current}$
\end{tabular}
Looking at this sub-table, we can base how to build the string. The current position at the start will be at position $[n,m]$ in the array. If the current spot we are at have equal letters at the array intersection of current (so at the beginning, it will be the $nth$ character of String A, and the $mth$ character of B), then we go to the diagonal position $x_{1}$ and add the letter of current to our string. We then repeat the process, making $x_{1}$ our new current, and the row above it $x_{2}$ and the column to the left $x_{3}$. If they are not the same letters, and $x_{2} < x_{3}$, choose $x_{2}$ as the new current, and add the letter that was in the row to the string. If it is none of the above cases, then make $x_{3}$ your new current and add the current column's letter to the beginning of your string.

The trace would look something like this: 

\begin{tabular}{| l | c | c | c | c | c | c | r | }
\hline
   	& * & \textit{z} & \textit{x} & \textit{y} & \textit{y} & \textit{z} & \textit{z} \\ \hline
  * 	& \cellcolor[gray]{0.8} Finish &   &   &   &   &   &   \\ \hline
\textit{z} & \cellcolor[gray]{0.8} add z &   &   &   &   &   &   \\ \hline
\textit{z} &   & \cellcolor[gray]{0.8} add z &   &   &   &   &   \\ \hline
\textit{y} &   & \cellcolor[gray]{0.8} add y &   &   &   &   &   \\ \hline
\textit{x} &   &   & \cellcolor[gray]{0.8} add x &   &   &   &   \\ \hline
\textit{z} &   &   & \cellcolor[gray]{0.8} add z & \cellcolor[gray]{0.8} add y &   &   &   \\ \hline
\textit{y} &   &   &   &   & \cellcolor[gray]{0.8}
 add y & \cellcolor[gray]{0.8} add z & \cellcolor[gray]{0.8} add z \\ \hline
\end{tabular}

The same table with the original values:

\begin{tabular}{| l | c | c | c | c | c | c | r | }
\hline
   	& * & \textit{z} & \textit{x} & \textit{y} & \textit{y} & \textit{z} & \textit{z} \\ \hline
  * 	   & \cellcolor[gray]{0.8} 0 & 1 & 2 & 3 & 4 & 5 & 6 \\ \hline
\textit{z} & \cellcolor[gray]{0.8} 1 & 1 & 2 & 3 & 4 & 5 & 6 \\ \hline
\textit{z} & 2 & \cellcolor[gray]{0.8} 2 & 3 & 4 & 5 & 5 & 6 \\ \hline
\textit{y} & 3 & \cellcolor[gray]{0.8} 3 & 4 & 4 & 5 & 6 & 7 \\ \hline
\textit{x} & 4 & 4 & \cellcolor[gray]{0.8} 4 & 5 & 6 & 7 & 8 \\ \hline
\textit{z} & 5 & 5 & \cellcolor[gray]{0.8} 5 & \cellcolor[gray]{0.8} 6 & 7 & 7 & 8 \\ \hline
\textit{y} & 6 & 6 & 6 & 6 & \cellcolor[gray]{0.8} 7 & \cellcolor[gray]{0.8} 8 & \cellcolor[gray]{0.8} 9 \\ \hline
\end{tabular}


\textit{An interesting little side note: you can derive the largest common substring of this problem by doing the same procedure as above, except only adding the letters when A[n]==B[m]}
\end{enumerate}
\end{enumerate}
\end{document}
