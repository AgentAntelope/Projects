\documentclass{article}
\usepackage[pdftex]{graphicx}
\usepackage{amsmath}

\pdfpagewidth 8.5in
\pdfpageheight 11in
\topmargin -1in
\headheight 0in
\headsep 0in
\textheight 8.5in
\textwidth 6.5in
\oddsidemargin 0in
\evensidemargin 0in
\headheight 77pt
\headsep 0in
\footskip .75in

\title{CS1510  Dynamic Programming Problems 24,\& 25, Reduction Problem 2}
\author{Rebecca Negley, Sean Myers}
\date{October 14, 2011}

\begin{document}
\maketitle

\begin{enumerate}
\setcounter{enumi}{23}
\item Give a polynomial time algorithm for the following problem. The input consists of a sequence 
\newline$R = R_0, ..., R_n$ of non negative integers, and an integer $k$. The number $R_i$ represents the number of
\newline users requesting some particular piece of information at time $i$ (say from a www.server. If the
\newline server broadcasts this information at some time $t$, the requests of all the users who requested the
\newline information strictly before time $t$ are satisfied. The server can broadcast this information at most 
\newline $k$ times.  The goal is to pick the $k$ times to broadcast in order to minimize the total time (over all
\newline requests) that requests/users have to wait in order to have their requests satisfied

Solution:

\item Assume that you are given a collection $B_1,...,B_n$ of boxes. You are told that the weight in kilograms
\newline of each box is an integer between 1 and some constant $L$, inclusive. However you do not know the
\newline specific weight of any box, and you do not know the specific value of $L$. You are also given a pan 
\newline balance. A pan balance functions in the following manner. You can give the pan balance any two
\newline disjoint sub-collections, say $S_1$ and $S_2$ of the boxes. Let $|S_1|$ and $|S_2|$ be the cumulative weight of the 
\newline boxes in $S_1$ and $S_2$, respectively. The pan balance then determines whether $|S_1| < |S_2|$, $|S_1| = |S_2|$,
\newline or $|S_1| > |S_2|$. You have nothing else at your disposal other than these $n$ boxes and the pan balance. 
\newline The problem is to determine if one can partition the boxes into two disjoint sub-collections of equal
\newline weight. Give an algorithm for this problem that makes at most $O(n^2L)$ uses of the pan balance. For
\newline partial credit, find an algorithm where the number of uses is polynomial in $n$ and $L$.

Solution:

%Reduction 2:
\item Show that if there is an $O(n^k)$, $k \ge 2$, time algorithm for inverting a nonsingular $n$ by $n$ matrix $C$
\newline then there is an $O(n^k)$ time algorithm for multiply two arbitrary $n$ by $n$ matrices $A$ and $B$.
\newline For a square matrix $A$, $A$ inverse, denoted $A^{-1}$, is the unique matrix such that $AA^{-1}= I$, where $I$ is 
\newline the identity matrix with 1's on the main diagonal and 0's every place else. Not that not every square
\newline matrix has an inverse, e.g. the all zero matrix.

Solution: If we can reduce multiplying two arbitrary n x n matrices to inverting a matrix, and the most inefficient part of the algorithm is the inversion, then the algorithm will run in at least $O(n^k)$. If we set up a matrix $C$, where the matrix of size 3n x 3n, where the matrix looks like:
\begin{tabular}{ |l  c  r |  }
I & A & 0  \\ 
0 & I & B  \\
0 & 0 & I  \\ 
\end{tabular}
(I being identity matrix). The setup of this matrix would take $9*n^2$, since it is trivial to set up an identity matrix, copy a matrix or 0-pad a matrix like so.

Then the inverse(derivation not shown) would look something like:
\begin{tabular}{ |l  c  r |  }
I & -A & A*B  \\ 
0 & I & -B  \\
0 & 0 & I  \\ 
\end{tabular}

Then all we would need to do is extrapolate the top right node (which takes $n^2$ amount of time), and we have our matrix multiplication.

The time to convert to the input matrix $C$ is $O(n^2)$, the inversion takes $O(n^k)$ where k must be greater than or equal to 2 and then once we have the output, the time to translate to the desired output is $O(n^2)$. Hence, the algorithm's slowest possible run time is that of the inverse multiplication $O(n^k)$.

\end{enumerate}
\end{document}

