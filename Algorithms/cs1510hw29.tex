\documentclass{article}
\usepackage[pdftex]{graphicx}
\usepackage{amsmath}

\pdfpagewidth 8.5in
\pdfpageheight 11in
\topmargin -1in
\headheight 0in
\headsep 0in
\textheight 8.5in
\textwidth 6.5in
\oddsidemargin 0in
\evensidemargin 0in
\headheight 77pt
\headsep 0in
\footskip .75in

\title{CS1510  Parallel Problems 15,17,18,20}
\author{Rebecca Negley, Sean Myers}
\date{November 14, 2011}

\begin{document}
\maketitle

\begin{enumerate}
%Problem 9
\setcounter{enumi}{15}
\item Give an algorithm for the minimum edit distance problem that runs in poly-log time on a CREW
\newline PRAM with a polynomial number of processors. Here poly-log means ($log^kn$) where $n$ is the 
\newline input size and $k$ is some constant independent of the input size.
\newline Recall that the input to this problem is a pair of strings $A = a_1 ...  a_m$ and $B = b_1 ... b_n$. The goal is
\newline to convert A into B as cheaply as possible. The rules are as follows. For a cost of 3 you can delete any
\newline letter. For a cost of 4 you can insert a letter in any position. For a cost of 5 you can replace any letter
\newline by any other letter
\newline Solution
\newline
\setcounter{enumi}{16}
\item Design a parallel algorithm that finds the maximum number in a sequence $x_1... x_n$ of (not necessarily
\newline distinct) integers. Your algorithm should run in time O($log log n$) on a CRCW PRAM with $n$ processors.
\newline
Solution:
\newline
\item Design a parallel algorithm that finds the maximum number in a sequence $x_1...x_n$ of (not necessarily
\newline distinct) integers in the range 1 to $n$. Your algorithm should run in constant time on a CRCW Priority
\newline PRAM with $n$ processors. Note that it is important here that the $x_i$'s have restricted range. In a
\newline CRCW priority PRAM, each processor has a unique positive integer identifier, and in the case of write
\newline conflicts, the value written is the value that the processor with the lowest identifier is trying to write.
\newline
Solution:
\newline
\setcounter{enumi}{19}
\item Show that if there is an an algorithm for a particular problem that runs in time $T(n, p)$ on a p
\newline processor CRCW machine, then there is an algorithm for this problem that runs in time $T(n, p)$ log p
on a p processor EREW machine.
\newline
Solution:
\end{enumerate}
\end{document}
