\documentclass{article}
\usepackage[pdftex]{graphicx}
\usepackage{amsmath}

\pdfpagewidth 8.5in
\pdfpageheight 11in
\topmargin -1in
\headheight 0in
\headsep 0in
\textheight 8.5in
\textwidth 6.5in
\oddsidemargin 0in
\evensidemargin 0in
\headheight 77pt
\headsep 0in
\footskip .75in

\title{CS1510  Reduction Problems 7, 9, \& 13}
\author{Rebecca Negley, Sean Myers}
\date{October 21, 2011}

\begin{document}
\maketitle

\begin{enumerate}
\setcounter{enumi}{6}
%Problem 7
\item Show that if one of the following three problems has a polynomial time algorithm, then they all do:
\begin{enumerate]
\item The input is two undirected graphs $G$ and $H$. The problem is to determine if the graphs are
\newline isomorphic.
\item The input is two directed graphs $G$ and $H$. The problem is to determine if the graphs are
\newline isomorphic.
\item The input is two undirected graphs $G$ and $H$ and an integer $k$. The problem is to determine if
\newline the graphs are isomorphic and all the vertices in each graph have degree $k$.
\end{enumerate}
Intuitively, two graphs are isomorphic if one can name/label the vertices so that the graphs are identical.
\newline More formally, two undirected graphs $G$ and $H$ are isomorphic if there is a bijection $f$ from the vertices
\newline of $G$ to the vertices of $H$ such that ($v,w$) is an edge in $G$ if and only if $(f(v), f(w))$ is an edge in $G$.
\newline More formally, two directed graphgs $G$ and $H$ are isomorphic if there is a bijection $f$ from the vertices
\newline of $G$ to the vertices of $h$ such that $(v,w)$ is a directed edge in $G$ if and only if ($f(v), f(w)$) is a directed
\newline edge in $G$. The degree of a vertex is the number of edges incident to that vertex.
\newline

Solution:

\setcounter{enumi}{8}
%Problem 9
\item The input to the Hamiltonian Cycle Problem is an undirected graph $G$. The problem is to find a
\newline Hamiltonian cycle, if one exists. A hamiltonian cycle is a simple cycle that spans $G$. Show that the
\newline Hamiltonian cycle problem is self reducible. That is, show that if there is a polynomial time algorithm
\newline that determines whether a graph has a Hamiltonian cycle, then there is a polynomial time algorithm to find Hamiltonian cycles. 
\newline
Solution:

\setcounter{enumi}{12}
\item Show that the following problem is $NP$-hard:
\newline INPUT: A graph $G$. Let $n$ be the number of vertices in $G$.
\newline OUTPUT: 1 if $G$ contains a simple cycle with \sqrt{$n$} edges, and 0 otherwise.
\newline Use the fact the following problem is $NP$-hard:
INPUT: A graph $G$.\newline
OUTPUT: 1 if $G$ contains a simple cycle that spans $g$, and 0 otherwise.

Solution:
\end{enumerate}
\end{document}
